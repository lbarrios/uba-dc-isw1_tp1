%--
%  Conclusiones: mencionar brevemente de qué formas les resultó más sencillo
%  encarar el TP. Por ejemplo en qué orden realizaron los diagramas, qué
%  aspectos presentaron las mayores dificultades, etc.
%--

\section{Conclusiones}

El presente trabajo resultó una efectiva introducción a varios de los problemas
que la \texttt{Ingeniería de Software} busca resolver.

El primer paso fue plantear los \texttt{fenómenos} que consideramos forman parte
de la modificación a la cadena \textbf{Mes\%}. Empezaron a surgir así los
distintos conceptos que luego se transformarían en \texttt{agentes}, junto con
las primeras \texttt{presunciones de dominio}.

Luego, nos abocamos al desarrollo de un \texttt{modelo de objetivos}. Durante el
transcurso de este proceso fue necesario \texttt{refinar} nuestros datos sobre
varios asuntos por lo que hubo una \texttt{interacción intensa} con el tutor,
que hacía las veces de CEO de la cadena. El resultado de esta etapa de modelado
quedó plasmado en un \texttt{diagrama de objetivos}.

Una vez que el diagrama de objetivos estuvo finalizado, esbozar el
\texttt{diagrama de contexto} y ejemplificar los potenciales \texttt{escenarios
de uso} fue relativamente sencillo.
