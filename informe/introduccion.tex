%-- 
%  Introducción: acá debería figurar todo lo necesario como para que un lector
%  no iniciado en el tema entienda el propósito general del sistema que van a
%  describir. Puede citar algunos fragmentos del enunciado si lo consideran
%  necesario, pero NO DEBE SER una copia del enunciado.
%--

\section{Introducción}

Mediante el presente trabajo, se nos pone frente a la tarea de realizar un
\textbf{análisis de requerimientos} y una \textbf{propuesta de un sistema de
software} para la cadena de supermercados MES\%, a pedido del CEO\footnote{El
tutor de este trabajo práctico, en este caso, cumple este rol.} de la compañía.
La propuesta debe ser planteada de tal forma que esta transmita claramente la
manera en que será cumplido cierto conjunto de objetivos, junto con los
mecanismos mediante los cuales el software se relacionará con los sistemas ya
existentes, y su beneficio directo y potencial.

A continuación, se incluye un resumen coloquial de los requerimientos, que
describe el contexto dentro del cual se solicitó la implementación del sistema.

\subsection{Contexto}

La cadena MES\% ha venido creciendo rápidamente desde sus comienzos, debido a
sus precios. Pero este crecimiento se ha detenido en los últimos seis meses. Se
han identificado dos razones por las cuales se presume que las ventas han
disminuído: 
\begin{enumerate}
  \item <<descontento de los clientes debido al excesivo tiempo de espera 
  en las colas de las cajas>>.
  \item <<ventas no concretadas por falta de mercadería>>.
\end{enumerate}
Con el objetivo de eliminar esos problemas, \fixme

% Completar aquí el contexto