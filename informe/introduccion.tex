%-- 
%  Introducción: acá debería figurar todo lo necesario como para que un lector
%  no iniciado en el tema entienda el propósito general del sistema que van a
%  describir. Puede citar algunos fragmentos del enunciado si lo consideran
%  necesario, pero NO DEBE SER una copia del enunciado.
%--

\section{Introducción}

Mediante el presente trabajo, se nos pone frente a la tarea de realizar un
análisis de requerimientos y una \textbf{propuesta de un sistema de software}
para la cadena de supermercados MES\%, a pedido del CEO\footnote{El tutor de
este trabajo práctico, en este caso, cumple este papel.} de la compañía. La
propuesta debe ser planteada de tal forma que esta transmita claramente la
manera en que será cumplido cierto conjunto de objetivos, junto con los
mecanismos mediante los cuales el software se relacionará con los sistemas ya
existentes, y su beneficio directo y potencial.

A continuación, se incluye un resumen coloquial de los requerimientos, que
describe el contexto dentro del cual se solicitó la implementación del sistema.

\subsection{Contexto}

% Completar aquí el contexto