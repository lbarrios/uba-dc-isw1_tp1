%-- 
%  Presunciones: acá deberían listar aquellas cuestiones que asumieron por
%  encima del enunciado. Estas cuestiones pueden provenir de alguna consulta con
%  docentes. También pueden provenir de alguna especulación o interpretación que
%  el grupo hizo.
%--

\section{Presunciones / Aclaraciones adicionales}
  
\begin{itemize}

  \item En caso de que la cadena de supermercados decida inaugurar más depósitos, tanto la reposición de stock como los traslados entre estos estarán a cargo de la compañía. A efectos prácticos, se asumirá, que existe un solo depósito, cuyo stock es la sumatoria del stock de todos los depósitos, y que todos los pedidos serán entregados desde el mismo.

  \item Una empresa de logística se encargará de todos los traslados que se originen en un depósito. El sistema deberá contar con una interfaz adecuada que actúe como canal de comunicación bilateral con dicha empresa. Este canal deberá ser apto tanto para realizar consultas y pedidos de forma online, a través de solicitudes directas al servidor de la compañía de logística, como para informar o ingresar datos de forma indirecta, a través de la impresión o la carga manual de: remitos de traslado, comprobantes de pago, facturas, hojas de ruta, etcétera.

  \item El sistema deberá llevar, en todo momento, un control activo del stock del depósito.

  \item Las entregas son realizadas por una empresa de logística externa al sistema. Asumimos que el sistema tiene una forma de calcular u obtener el tiempo mínimo de entrega, de forma tal de poder ofrecerle al cliente un rango de fechas de entrega válido.

  \item Existe un Dpto. de stock, que se encarga de organizar los pedidos a las proveedoras. 

  \item No puede permanecer un pedido abierto hasta el momento que sale el camion. Se cierra al momento que el Depósito informa que el pedido fue armado. 

  \item Las entregas son realizadas por una empresa de logística externa al sistema. Asumimos que el sistema tiene una forma de calcular u obtener el tiempo mínimo de entrega, de forma tal de poder ofrecerle al cliente un rango de fechas de entrega válido.

\end{itemize}
