% ------ headers globales -------------
\documentclass[11pt, a4paper, twoside]{article}
\usepackage{header}
\usepackage{config}
% -------------------------------------
\begin{document}

% -- Carátula --
\clearpage{\pagestyle{empty}\input{caratula}}

%-- Índice --
\clearpage{%
  \pagestyle{empty}\tableofcontents%
  \vspace{3cm}%
  \cleardoublepage%
}

%-- A partir de aquí, pongo el contador de páginas en 1 --
\setcounter{page}{1}

%--
%-- Introducción
%--
\section{Introducción}

  %--
  %-- Introducción: Objetivo del Trabajo Práctico
  %--
  \subsection{Objetivos del Trabajo Práctico}

Mediante el presente trabajo, se nos pone frente a la tarea de realizar un
análisis de requerimientos y una \textbf{propuesta de un sistema de software}
para la cadena de supermercados MES\%, a pedido del CEO de la compañía. La
propuesta debe ser planteada de tal forma que esta transmita claramente la
manera en que será cumplido cierto conjunto de objetivos, junto con los
mecanismos mediante los cuales el software se relacionará con los sistemas ya
existentes, y su beneficio directo y potencial.

A continuación, se transcribe el texto original que describe el contexto dentro
del cual se implementará el sistema, y posteriormente a este se incluyen las
presunciones y aclaraciones adicionales, extraídas a partir de nuestra
interpretación del susodicho, o bien discutidas con el CEO\footnote{El tutor de
este trabajo práctico, en este caso, cumple este papel.} de la compañía.

  %--
  %-- Introducción: Contexto
  %--
  \subsection{Contexto}

  En los últimos años, la cadena de supermercados Mes\% viene creciendo
rápidamente y esto se debe, principalmente, a que tienen los mejores precios de
toda la ciudad.

  Sin embargo, según los estudios entregados por una consultora privada, en los
últimos seis meses las ventas cayeron casi un 15\%. La razón de esta baja tiene
dos motivos: Por un lado, los locales tienen cada vez una mayor cantidad de
clientes, generando molestias debido al excesivo tiempo de espera que se produce
en las colas de las cajas. Por otro lado, en muchas ocasiones ocurre que las
ventas no pueden efectuarse dada la falta de mercadería disponible en las
góndolas.

El CEO de la compañía ha decidido tomar cartas en el asunto e impulsar el
desarrollo de un Software que estimule a sus clientes a efectuar sus compras
desde cualquier dispositivo con internet. La idea central es que las compras
puedan realizarse desde la web \textit{www.mesporciento.com} (se pretende que el
diseño de interfaz de usuario sea \textbf{responsive} de manera tal que sea
usable tanto en un navegador estándar como en los dispositivos móviles). Para
esto, los usuarios deberían registrarse (una única vez). Luego, a partir de ese
momento, podrán efectuar pedidos acordando con nosotros la fecha de entrega. Las
entregas podrán realizarse a cualquier punto de la ciudad. Para esto, Mes\%
contaráa con camionetas que se encargarán de distribuir los pedidos una vez que
fueron cerrados.

Actualmente, Mes\% cuenta con cinco locales y un depósito. El CEO plantea como
estrategia de crecimiento de la compañía el aumento del tamaño y/o de la
cantidad de depósitos y no así de la cantidad de locales, ya que la mayor
cantidad de ventas deberían provenir de las ventas online.

Ustedes han sido seleccionados para llevar a cabo el desarrollo del sistema. En
esta primera fase, su tarea es obtener y documentar los requerimientos, que
serán provistos por el CEO de la cadena. En algunas reuniones preliminares, nos
transmitió algunas inquietudes y opiniones que se fueron relevando a clientes,
responsables de stock, encargado de entregas y encargado de armado de pedidos:

\begin{itemize}

\item ``No podemos tener problemas de stock cuando un pedido se realiza de forma
online.''

\item ``Aprovechando la construcción del software deberíamos, desde cualquier
local, solicitar la reposición de stock.''

\item ``Si bien, asumimos que nuestros clientes al momento de la fecha de
entrega van a estar en los domicilios, ¿qué va a pasar cuando no estén? En
muchos casos puede ser que sea por olvido, pero en muchos otros no. Tenemos que
poder actuar ante esta situación.''

\item ``Un cliente no puede realizar un pedido si aún tiene un pedido no
entregado. Sin embargo, si el pedido está aún pendiente de preparación, debería
poder modificarlo.''

\item ``Los pedidos sólo serán entregados si están pagos. El pago puede hacerse
al momento de hacer el pedido o contra entrega`.''

\item ``Como dueños del Sistema, queremos contar con información que sea
interesante como para tomar mejores decisiones a futuro.''

\end{itemize}

Se espera, además, que el sistema escale y sea seguro.

  %--
  %-- Introducción: Presunciones / Aclaraciones adicionales
  %--
  \subsection{Presunciones / Aclaraciones adicionales}
  
  \begin{itemize}

  \item En caso de que la cadena de supermercados decida inaugurar más
depósitos, tanto la reposición de stock como los traslados entre estos estarán a
cargo de la compañía. A efectos prácticos, \textbf{se permite, y se asumirá, que
existe un solo depósito}, cuyo stock es la sumatoria del stock de todos los
depósitos, y que todos los pedidos serán entregados desde el mismo.

  \item Una empresa de logística se encargará de todos los traslados que se
originen en un depósito. El sistema deberá contar con una \textbf{interfaz
adecuada que actúe como canal de comunicación bilateral} con dicha empresa. Este
canal deberá ser apto tanto para \textbf{realizar consultas y pedidos de forma
online}, a través de solicitudes directas al servidor de la compañía de
logística, como para \textbf{informar o ingresar datos de forma indirecta}, a
través de la impresión o la carga manual de: remitos de traslado, comprobantes
de pago, facturas, hojas de ruta, etcétera.

  \item El sistema deberá llevar un \textbf{control activo del stock del
depósito}.


  \end{itemize}

\newpage

\fixme

\newpage
%--
%-- Escenarios Hipotéticos
%--
\section{Escenarios Hipotéticos}

\fixme

\newpage
%--
%-- Desarrollo
%--
\section{Desarrollo}

\fixme

\newpage

  %--
  %-- Diagrama de Contexto
  %--

  \subsection{Diagrama de Contexto}

  \fixme

  \newpage

  %--
  %-- Modelo de Objetivos
  %--

  \subsection{Modelo de Objetivos}

  \fixme

  \newpage

\end{document}
